\documentclass[sigconf]{acmart}

\input{format/final}

\begin{document}
\title{Analyzing everyday challenges of people with visual impairments}


\author{Tousif Ahmed}
\orcid{HID237}
\affiliation{%
\institution{Indiana University}
  \streetaddress{150 S Woodlawn Avenue}
  \city{Bloomington} 
  \state{Indiana} 
  \postcode{47405}
}
\email{touahmed@indiana.edu}



\begin{abstract}
       People with visual impairments face varieties of problem in their daily lives. Nowadays, modern technology especially camera-based technologies are helping people with visual impairment in their everyday tasks ranging from daily household activity to navigation. Users are using camera based applications where they are sharing photos and asking questions. Based on the asked question and shared photo, automated tools or human crowd workers are helping the visually impaired people in their tasks. By exploring the questions, it is possible to understand the problems and challenges of people with visual impairments. However, the volume of such data makes it impossible to analyze the questions manually.  Big data analytics could help us to understand the challenges of people with visual impairments. To understand the challenges, we analyzed the VizWiz data set which contains more than 33,500 questions asked by people with visual impairments. In this paper, we report on the data and shed light on the challenges.  

\end{abstract}

\keywords{E534, HID 237,  Big Data, Accessibility Issues, People with Visual Impairments}


\maketitle


\section{Introduction}
People with visual impairments face a variety of problems in their daily lives and need assistance. They need assistance with detecting objects, identifying money, navigation, transportation, household activities, cooking, and various other activities. Sighted person on rely on vision on so many things that it is almost impossible to visualize and understand the problems of people with visual impairments. Although there are variety of tools available to simulate the challenges and experiences of people with visual impairments, it is very  





\begin{acks}

The authors would like to thank Professor Gregor von Laszewski for helping us with the instruction and resources that were required to complete this paper. We would also to like to thank the associate instructors for being available on the course website all the time and helping us with their answers.

\end{acks}



\bibliographystyle{ACM-Reference-Format}
\bibliography{report} 
\newpage
\appendix

%We include an appendix with common issues that we see when students submit papers. One particular important issue is not to use the underscore in bibtex labels. Sharelatex allows this, but the proceedings script we have does not allow this.

W%hen you submit the paper you need to address each of the items in the
%issues.tex file and verify that you have done them. Please do this
%only at the end once you have finished writing the paper. To d this
%cange TODO with DONE. However if you check something on with DONE, but
%we find you actually have not executed it correcty, you will receive
%point deductions. Thus it is important to do this correctly and not
%just 5 minutes before the deadline. It is better to do a late
%submission than doing the check in haste. 

\section{Issues}

\TODO{A format issue cause another one more page}







\end{document}
